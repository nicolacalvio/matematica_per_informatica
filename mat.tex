
\documentclass{article}
\usepackage{amsmath} % for math symbols and equations

\title{Metodi matematici per l'informatica}
\author{Nicola Calvio}

\begin{document}

\maketitle

\section{Introduzione}

Appunti del corso di Metodi matematici per l'informatica, tenuto dal prof.~Cenciarelli presso l'università Sapienza di Roma.

\section{Insiemi}

\subsection{Proprietà degli insiemi}
%elenco puntato
\begin{itemize}
    \item due insiemi sono uguali se hanno gli stessi elementi.
    \item tra tutti i sottoinsiemi di un insieme troviamo sempre l'insieme vuoto e l'insieme stesso.
\end{itemize}

\subsection{Operazioni tra insiemi}
\begin{itemize}
    \item unione: $A \cup B = \{x | x \in A \lor x \in B\}$ è l'insieme al quale appartengono gli elementi che appartengono almeno a uno degli insiemi
    \item intersezione: $A \cap B = \{x | x \in A \land x \in B\}$ è l'insieme al quale appartengono gli elementi che appartengono ad entrambi gli insiemi
    \item differenza: $A - B = \{x | x \in A \land x \notin B\}$ è l'insieme al quale appartengono gli elementi che appartengono ad $A$ ma non a $B$
    \item complementare: $A^c = \{x | x \notin A\}$
\end{itemize}

\subsection{Operazioni su insiemi}
\begin{itemize}
    \item potenza: $P(A) = \{X | X \subseteq A\}$ è l'insieme di tutti i sottoinsiemi di $A$ (compreso l'insieme vuoto e l'insieme stesso)
\end{itemize}

\subsection{Coppie ordinate}
\begin{equation}
    (a,b) \not= (b,a) %newline
\end{equation}
\begin{equation}    
    \{a,b\} = \{b,a\}
\end{equation}
\begin{equation}    
    (a,b) = \{a,\{a,b\}\}
\end{equation}

%proprietà delle operazioni tra insiemi
\subsection{Proprietà delle operazioni tra insiemi}
\begin{itemize}
    \item commutativa: $A \cup B = B \cup A$ e $A \cap B = B \cap A$
    \item associativa: $(A \cup B) \cup C = A \cup (B \cup C)$ e $(A \cap B) \cap C = A \cap (B \cap C)$
    \item distributiva: $A \cup (B \cap C) = (A \cup B) \cap (A \cup C)$ e $A \cap (B \cup C) = (A \cap B) \cup (A \cap C)$
    \item leggi di De Morgan: $(A \cup B)^c = A^c \cap B^c$ e $(A \cap B)^c = A^c \cup B^c$
\end{itemize}

\subsection{Prodotto cartesiano di due insiemi}
%introduzione sulle coppie ordinate
Una coppia ordinata è un insieme di due elementi, in cui l'ordine degli elementi è importante. Due coppie ordinate sono uguali se e solo se i loro elementi sono uguali e sono in corrispondenza uno a uno.\\

%prodotto cartesiano
Il prodotto cartesiano di due insiemi $A$ e $B$ è l'insieme 
\begin{equation}
    A \times B = \{(a,b) | a \in A \land b \in B\}
\end{equation}
%esempio
\begin{equation}
    A = \{1,2\} \quad B = \{a,b\} \quad A \times B = \{(1,a),(1,b),(2,a),(2,b)\}
\end{equation}

\subsection{Relazioni}
%bold text
\subsubsection{Relazioni binarie}
scriviamo $a R b$ per indicare che $(a,b) \in R$ 

\subsubsection{Proprietà delle relazioni} 
\begin{itemize}
\item $R \subseteq A \times A$  è antiriflessiva se per ogni $a \in A, (a,a) \notin R$
\item $R \subseteq A \times A$  è riflessiva se per ogni $a \in A, (a,a) \in R$
\item 
\item $R \subseteq A \times A$  è simmetrica se per ogni $a,b \in A, a R b \Rightarrow b R a$
\end{itemize}

\section{Conclusion}

This is the conclusion section of your math document.

\end{document}
